\section{\textit{Insertion Sort}}

\subsection{Explicação do Algoritmo}
A melhor analogia a ser fazer quando falamos de \textit{Insertion Sort} é um jogo de cartas. Você começa com a mão vazia e, conforme for recebendo as cartas, você as adiciona
\textbf{em ordem} na sua mão, comparando essa nova carta com as que já estão na sua mão. Nós vamos iterar o conjunto de dados (vetor) e vamos comparar com os elementos anteriores.
Se ele for menor, nós vamos dar um \textit{shift} em todos os elementos a partir da posição onde queremos colocar esse elemento.

\subsection{Implementação Iterativa}
\begin{lstlisting}[language=C]
void insercaoIterativo(int* vector, int numberOfElements){
    int lixo = -1;
    int n = 2;
    while(n <= numberOfElements){
        int aux = vector[n - 1];
        vector[n - 1] = lixo;
        int i = n - 2;
        print(vector, n);
        while(i >= 0 && vector[i] > aux) {
            vector[i + 1] = vector[i];
            vector[i] = lixo;
            i--;
            print(vector, n);
        }
        vector[i + 1] = aux;
        print(vector, n);
        n++;
    }
}

\end{lstlisting}

\subsection{Implementação Recursiva}
\begin{lstlisting}[language=C]
void insercaoRecursivo(int* vector, int numberOfElements){
    int lixo = -1;
    if(numberOfElements < 2)
        return;

    insercaoRecursivo(vector, numberOfElements - 1)
    int aux = vector[n - 1];
    int n = 2;
    int i = n - 2;
    print(vector, n);
    
    while(i >= 0 && vector[i] > aux) {
        vector[i + 1] = vector[i];
        vector[i] = lixo;
        i--;
        print(vector, n);
    }
    vector[i + 1] = aux;
    print(vector, n);
}
\end{lstlisting}
